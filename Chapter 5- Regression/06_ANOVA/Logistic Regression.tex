%%%%%%%%%%%%%%%%%%%%%%%%%%%%%%%%%%%%%%%%%%%%%%%%%%%%%%%%%%%%%
%%%%% Sample examination layout for 		       	%%%%%
%%%%%          dit_maths_exam.sty 			%%%%%
%%%%%							%%%%%
%%%%%	V3 September 2015    				%%%%%
%%%%%	- Use new sty file to mirror CoSH template 	%%%%%
%%%%%   - include bw option for black & white logo      %%%%%
%%%%%	- instructions on new programmes		%%%%%
%%%%%							%%%%%
%%%%%	V2.1 October 2014    				%%%%%
%%%%%	- Fix bug using old sty file naming convention 	%%%%%
%%%%%							%%%%%
%%%%%	V2 October 2014    				%%%%%
%%%%%	- Remove reference to Springboard programmes 	%%%%%
%%%%%	- included new external examiners		%%%%%
%%%%%							%%%%%
%%%%%	V1.1 October 2013    				%%%%%
%%%%%	- New external examiner instruction included	%%%%%
%%%%%	- Two new rubrics: DT205/DT220/1 Programming	%%%%%
%%%%%							%%%%%
%%%%%	V1 February 2013    				%%%%%
%%%%%	- Original version				%%%%%
%%%%%%%%%%%%%%%%%%%%%%%%%%%%%%%%%%%%%%%%%%%%%%%%%%%%%%%%%%%%%

\documentclass[a4paper,12pt]{article}

%\usepackage{amsmath,color,graphicx,epstopdf,wrapfig,enumerate}
\usepackage{amsmath,enumerate,graphics,graphicx,color,amsthm}
%\usepackage{venturis2}
%\usepackage[T1]{fontenc}
\usepackage[adobe-utopia]{mathdesign}
\usepackage[T1]{fontenc}
\usepackage{enumitem}
\newcommand{\lqs}{\leqslant}
\newcommand{\gqs}{\geqslant}
\newcommand{\R}{\mathbb{R}}
\definecolor{gray}{RGB}{128,128,128}
\definecolor{lime}{RGB}{153,204,0}
\newtheoremstyle{Qstyle}{}{20pt}{}{}{\bfseries}{:\quad}{ }{}
\theoremstyle{Qstyle}
\newtheorem{q}{Q\hspace{-2pt}}
%%%%%%%%%%%%%%%%%%%%%%%%%%%%%%%%%%%%%%%%%%%%%%%%%%%%%%%%%%%%%
%%%%% 	ENSURE LATEST VERSION OF STYLE IS USED HERE	%%%%%
%%%%%   ---use option argument bw to use b&w logo       %%%%%
%%%%%%%%%%%%%%%%%%%%%%%%%%%%%%%%%%%%%%%%%%%%%%%%%%%%%%%%%%%%%

%\usepackage{dit_maths_exam_V3}		%use for color logo
\usepackage[bw]{dit_maths_exam_V3}	%use for bw logo

%%%%%%%%%%%%%%%%%%%%%%%%%%%%%%%%%%%%%%%%%%%%%%%%%%%%%%%%%%%%%
%%%%% include optional packages here 		       	%%%%%
%%%%%%%%%%%%%%%%%%%%%%%%%%%%%%%%%%%%%%%%%%%%%%%%%%%%%%%%%%%%%
\usepackage{}

%%%%%%%%%%%%%%%%%%%%%%%%%%%%%%%%%%%%%%%%%%%%%%%%%%%%%%%%%%%%%
%%%%% include local command definitions here	       	%%%%%
%%%%%%%%%%%%%%%%%%%%%%%%%%%%%%%%%%%%%%%%%%%%%%%%%%%%%%%%%%%%%


%%%%%%%%%%%%%%%%%%%%%%%%%%%%%%%%%%%%%%%%%%%%%%%%%%%%%%%%%%%%%
%%%%% enter examination code from timetable	   	%%%%%
%%%%% ---ensure all concurrent exams are included---	%%%%% 
%%%%%%%%%%%%%%%%%%%%%%%%%%%%%%%%%%%%%%%%%%%%%%%%%%%%%%%%%%%%%
\renewcommand{\code}{W8998/101}


%%%%%%%%%%%%%%%%%%%%%%%%%%%%%%%%%%%%%%%%%%%%%%%%%%%%%%%%%%%%%
%%%%% enter module title including module code 	       	%%%%%
%%%%% ---format: MATH XXXX: Title of Module		%%%%%
%%%%%%%%%%%%%%%%%%%%%%%%%%%%%%%%%%%%%%%%%%%%%%%%%%%%%%%%%%%%%
\renewcommand{\mtitle}{MATH 4001: Probability and Statistical Inference}

%%%%%%%%%%%%%%%%%%%%%%%%%%%%%%%%%%%%%%%%%%%%%%%%%%%%%%%%%%%%%
%%%%% enter all programmes 				%%%%%
%%%%% ---separate with \\[10pt] & use the definitions:	%%%%%
%%%%%		\dt{205} \dt{220} 			%%%%%
%%%%%		\dt{6248} \dt{7248} \dt{8248} 		%%%%%
%%%%%		\dt{8998}				%%%%%
%%%%%		\dt{9205} (includes DT9206)		%%%%%
%%%%%		\dt{9209} (includes DT9210)		%%%%%
%%%%%		\dt{9211} (includes DT9212)		%%%%%
%%%%%[only for legacy/repeat students 2015/2016] 	%%%%%
%%%%%		\dt{234} \dt{238} 			%%%%%
%%%%%%%%%%%%%%%%%%%%%%%%%%%%%%%%%%%%%%%%%%%%%%%%%%%%%%%%%%%%%
\renewcommand{\progs}{\dt{8998} \\[10pt] } 

%%%%%%%%%%%%%%%%%%%%%%%%%%%%%%%%%%%%%%%%%%%%%%%%%%%%%%%%%%%%%
%%%%% enter stage number		 	       	%%%%%
%%%%%  	- use second definition if no stage  		%%%%%
%%%%%%%%%%%%%%%%%%%%%%%%%%%%%%%%%%%%%%%%%%%%%%%%%%%%%%%%%%%%%
%\renewcommand{\stage}{Stage 1} 	%comment if no stage
\renewcommand{\stage}{\vspace{-0.25in}}	%comment if stage exists

%%%%%%%%%%%%%%%%%%%%%%%%%%%%%%%%%%%%%%%%%%%%%%%%%%%%%%%%%%%%%
%%%%% enter examination session	& ACADEMIC year		%%%%%
%%%%% ---use following definitions:			%%%%%
%%%%%		\winter \summer \autumn			%%%%%
%%%%% ---use format YYYY/YYYY				%%%%%
%%%%%%%%%%%%%%%%%%%%%%%%%%%%%%%%%%%%%%%%%%%%%%%%%%%%%%%%%%%%%
\renewcommand{\session}{\winter}
\renewcommand{\acyear}{2015/2016}

%%%%%%%%%%%%%%%%%%%%%%%%%%%%%%%%%%%%%%%%%%%%%%%%%%%%%%%%%%%%%
%%%%% enter examiners					%%%%%
%%%%% ---use following definitions:			%%%%%
%%%%%		\appleby \murphy \riordan \tuite	%%%%%
%%%%% Head of School will be included automatically	%%%%%
%%%%%%%%%%%%%%%%%%%%%%%%%%%%%%%%%%%%%%%%%%%%%%%%%%%%%%%%%%%%%
\renewcommand{\examiner}{Dr. John S. Butler}
%\renewcommand{\external}{\riordan}  	%comment if no external
\renewcommand{\external}{ } 		%comment if external exists

%%%%%%%%%%%%%%%%%%%%%%%%%%%%%%%%%%%%%%%%%%%%%%%%%%%%%%%%%%%%%
%%%%% enter date time					%%%%%
%%%%% ---use following formats:				%%%%%
%%%%%	Thursday, 22 August 2014	2.00 -- 4.00 pm	%%%%%
%%%%%%%%%%%%%%%%%%%%%%%%%%%%%%%%%%%%%%%%%%%%%%%%%%%%%%%%%%%%%
\renewcommand{\exdate}{Monday, 4th January 2016}
\renewcommand{\extime}{9.30 -- 11.30 am}
%\newcommand{\duration}{2 hours}
%\newcommand{\duration}{3 hours}
\newcommand{\duration}{2 hour}
\newcommand*{\Comb}[2]{{}^{#1}C_{#2}}%
\newcommand*{\Pick}[2]{{}^{#1}P_{#2}}%
%%%%%%%%%%%%%%%%%%%%%%%%%%%%%%%%%%%%%%%%%%%%%%%%%%%%%%%%%%%%%
%%%%% rubrics: comment/uncomment as necessary		%%%%%
%%%%% ---there are two rubric lines rubrica & rubricb	%%%%%
%%%%% ---use standard wordings given			%%%%%
%%%%%%%%%%%%%%%%%%%%%%%%%%%%%%%%%%%%%%%%%%%%%%%%%%%%%%%%%%%%%
%\renewcommand{\rubrica}{Attempt three questions only}
%\renewcommand{\rubrica}{Full marks may be obtained by answering three questions.  Candidate's three best questions will contribute to their final mark.
%\renewcommand{\rubrica}{Attempt all qustions}
%\renewcommand{\rubrica}{Attempt question 1 and any two other questions}
\renewcommand{\rubrica}{Answer four questions}
%\renewcommand{\rubrica}{Attempt question 1 and any three other questions}
\renewcommand{\rubricb}{All questions carry equal marks}
%\renewcommand{\rubricb}{Answer at least two questions from each section.  All questions carrry equal marks}
%\renewcommand{\rubricb}{Question 1 carries 40 marks.  All other questions carry 20 marks each.}

%%%%%%%%%%%%%%%%%%%%%%%%%%%%%%%%%%%%%%%%%%%%%%%%%%%%%%%%%%%%%
%%%%% comment out those that are not permitted		%%%%%
%%%%% PLEASE ONLY INCLUDE ITEMS THAT ARE NECESSARY 	%%%%%
%%%%% AND APPROPRIATE					%%%%%
%%%%%%%%%%%%%%%%%%%%%%%%%%%%%%%%%%%%%%%%%%%%%%%%%%%%%%%%%%%%%
\calculators	%comment out if not permitted
%\tables		%comment out if Mathematical tables not to be provided
\stats		%comment out if New Cambridge Statistical Tables not to be provided



	\begin{document}
	\thispagestyle{empty}
	\begin{center}
	\Large{\bf{ Logistic Regression}}
	\end{center}
	\begin{enumerate}
		
		\item
		
		The NFL collected data on field goal success in professional American Football games. The data consisted of a binary variable for success (Success=1, Failure=0) and the distance in yards from the goal. The data were submitted to a logistic regression, a portion of the R out of the analysis is given below\\
			\noindent\fbox{%
				\parbox{340pt}{%
					Call:
					glm(formula = Success $\sim$ Yards, family = binomial("logit"), data = fieldgoal)\\
					
					Deviance Residuals: \\
					\begin{tabular}{ccccc}
						Min    &     1Q &    Median &        3Q  &      Max \\  
						-2.6568  &  0.2718 &   0.4166 &   0.6938  &  1.4750  \\  
					\end{tabular}\\
					Coefficients:\\
					\begin{tabular}{lrr}
						& Estimate & Std. Error  \\
						(Intercept) &5.69788  &  0.45110  \\
						Yards        &  -0.10991 &   0.01058     
					\end{tabular}
				}
			}
			\begin{enumerate}
				\item Write the formula of the logistic regression, explaining the terms used in the expression.
				\mrk{3 marks}
				\item Calculate the z-value of the estimates
				\mrk{5 marks}
				\item What are the estimated odds that a 60 yard field goal will succeed
				\mrk{4 marks}
				\item What is the probability that a 30 yard field goal will succeed, and what is probability it will fail. 
				\mrk{5 marks}
				
			\end{enumerate}
		
		\newpage
			\item  Sahoo and Pandalai (1999) conducted a study on the success or failure of finding gold deposits as a function of water/chemical factors: As level, Sb level, and presence==1 or absence==0 of lineament. A portion of the R analysis is given below\\
			\noindent\fbox{%
				\parbox{340pt}{%
					Call:
					glm(formula = Presence $\sim$ Aslevel+Sblevel+LineamentProx, family = binomial("logit"), data = gold)\\
					
					Deviance Residuals: \\
					\begin{tabular}{ccccc}
						Min    &     1Q &    Median &        3Q  &      Max \\  
						-2.6568  &  0.2718 &   0.4166 &   0.6938  &  1.4750  \\  
					\end{tabular}\\
					Coefficients:\\
					\begin{tabular}{lrr}
						& Estimate & Std. Error  \\
						(Intercept) &-7.6096  &   3.1661   \\
						Aslevel        & 1.2046  &   0.4899    \\
						Sblevel        & 1.4210    & 0.7301   \\    
						LineamentProx  & 3.1973    & 1.8911   
					\end{tabular}
				}
			}
			\begin{enumerate}
				\item Write the formula of the logistic regression, explaining the terms used in the expression.
				\mrk{3 marks}
				\item Calculate the z-value of the estimates.
				\mrk{7 marks}
				\item What are the probability of finding a gold deposit given an As level of 3, Sblevel of 4 and no lineament.
				\mrk{4 marks}
				\item Given a 95\% rejection criteria with a z-criteria of (-1.96, 196), are the estimates of the coefficents significant. 
				\mrk{3 marks}
				
			\end{enumerate}
			
		
		
		
	\end{enumerate}
	
	
	\newpage
	
	\section*{Formulae}
	
	\subsubsection*{Counting Rules}
	\subsubsection*{Permutations }
	\[\Pick{n}{r}=\frac{n!}{(n-r)!}\] 
	\subsubsection*{Combinations}
	\[\Comb{n}{r}=\left( \begin{array}{c}
	n  \\
	r  \\
	\end{array} \right)=\frac{n!}{r!(n-r)!}\] 
	
	\subsubsection*{Addition of Probability}
	\[p(A\cup B)=p(A)+p(B)-p(A\cap B)\]
	\subsubsection*{Conditional Probability}
	\[p(A|B)=\frac{p(A \cap B)}{p(B)}\]
	\subsubsection*{Bayes Formula}
	\[p(H_1|A)=\frac{p(A|H_1)p(H_1)}{p(A|H_1)p(H_1)+p(A|H_2)p(H_2)}\]
	
	
	\subsubsection*{Probability Mass Functions}
	\[E[X]=\Sigma_{i=1} x_i p(x_i)\]
	\[Var[X]=\Sigma_{i=1} (x_i-\mu)^2 p(x_i)\]
	\subsubsection*{Geometric Distributions}
	\[p(k)=q^{(k-1)}p, \ \ k=1,2,... \]
	\[E[k]=\frac{1}{p} \ \ \ Var[k]=\frac{q}{p^2}\]
	
	\subsubsection*{Binomial Distributions}
	\[p(k)= \left( \begin{array}{c}
	n  \\
	k  \\
	\end{array} \right)p^kq^{n-k}, \ \ k=0,1,2,...n \]
	\[E[k]=np\ \ \ Var[k]=npq\]
	
	\subsubsection*{Poisson Distributions}
	\[p(k)=\frac{\lambda^ke^{-\lambda}}{k!}, \ \ k=0,1,2,... \]
	\[E[k]=\lambda \ \ \ Var[k]=\lambda \]
	
	\subsubsection*{Chi squared}
	\[\chi_{GoF}=\sum \frac{(O-E)^2}{E} ~\chi^2_{k-1},\ \  \chi_{Ind}=\sum \frac{(O-E)^2}{E} ~\chi^2_{(r-1)(c-1)} \]
	
	\subsubsection*{Linear Regression}
	\[  y_i=\beta_0+\beta_1 x_{i1} +\beta_2 x_{i2}+... \]
	\[  y_i=\beta_0+\beta_1 x_{i1}\]
	\[  \beta_1 =\frac{Cov(x,y)}{SS_{xx}}, \ \  \beta_0 =\bar{y}-\beta_1\bar{x}\]
	
	\subsubsection*{Logistic Regression}
	\[  p_i=\frac{e^{\eta_i}}{1+e^{\eta_i}}, \ \ \ \eta_i=\beta_0+\beta_1 x_{i1} +\beta_2 x_{i2}+... \]
	
	\end{document}